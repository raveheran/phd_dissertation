\chapter*{General Discussion}
\label{chap:general_discussion}

\addcontentsline{toc}{chapter}{General Discussion}

\acresetall

this is general discussion

overarching terms \enquote{accommodation}



\todo[inline]{a part about evaluating accommodation: people are usually not aware of accommodation happening, and if they do, they can't evaluate it in a consistent and reliable way, as it is not necessarily related to measurable features in the acoustic signal}

\citet{Babel2012role} % she says that accommodation evaluation by humans is "holistic" and does not necessarily correlate with specific features in the signal


\citet{Levitan2016implementing} % that's the paper about Levitan's system that entrains in a relatively simple matter. take this as as a baseline when talking about sophistication of system behavior

\todo[inline]{somewhere say that temporal and mutual aspects are emphasized and are usually not in the focus of accommodation research}

%%%%%%%%%%%%%%%%%%%%
% taken from System chapter. use this after talking about what the system can do and maybe some limitations, then this is potential extensions / directions for improvement
Future work will pursue two independent directions:
regarding phonetic convergence, supporting more features will make the system more comprehensive and useful for studying a wider range of phenomena.
Specifically, adding support for supra-segmental (i.e., prosodic) features will enable replication of experiments similar to, e.g., \citet{Levitan2014acoustic, Levitan2016implementing} in the same manner as in \cref{sec:showcase}.

Regarding user acceptance, it would be interesting to examine whether users show any preference toward an \ac{sds} that converges to their speech on the phonetic level, and whether they would change their speaking style based on the system's output, forming an interaction with mutual and dynamic convergence.
The first research question can be tested by comparing user interaction with a baseline system and one with convergence capabilities, and evaluating the users' performance and satisfaction.
The second research question can be investigated by comparing the users' speech when interacting with either system configuration.
Additionally, to test the system's influence on users' speech, the users can train with an intelligent \acf{call} or \acf{capt} system, which will change its learner model based on their input.
Task completion rate, performance accuracy, and completion time metrics can be used to evaluate how helpful the system is.
%%%%%%%%%%%%%%%%%%%%

\todo[inline]{start with saying that accommodation in HCI is still a relatively new research area and there are no common methods for measuring and evaluating this. reasons -- maybe because there is no defined measuring unit, specifically when is comes to changes overtime and not only point vs. point. also, since there is no "correct" and "incorrect" answers, it is hard to compare against some gold standard and say whether results are good.}

\todo[inline]{having common terminology can be a first step toward comparable and common research methods. part of this work is to define some of the terms used to describe the various types of accommodation that could occur and the relations between them (cref).}




\todo[inline]{here do the connection that changes are something that can be better observed over time -- and then go into details why, say that it's one of the main points on this thesis and give examples how higher temporal resolution grants better look into the nature of changes and more interesting insights}






One of the main goals of this thesis is to depict vocal accommodation as one comprehensive process that includes multiple related parts, from examining effects in \ac{hhi}, via approaches to model them for computers, and of course the technical aspects of integrating and simulating them in \acp{sds} (see \cref{fig:roadmap_adaptive_sds}).
To get a better understanding of accommodation in \ac{hci}, it is important to see the connection between these parts and not investigate them in isolation from one another.
More often than not, the technical side of spoken \ac{hci} (e.g., \ac{asr} accuracy or \ac{tts} quality) are designed and developed separately from the user perspective.
While this is understandable when aiming purely at performance improvements, it is problematic when addressing dialogue-related problems.
However, since accommodation in \ac{hci} involves both computers and humans, both sides should be considered in the research and development of such systems.
Ultimately, they need to work together to achieve better communication, just like human interlocutors in \ac{hhi}.
In the case of accommodation, investigating effects in humans that are not relevant or cannot be implemented in computers doesn't contribute to the advancement toward accommodative systems.
Similarly, accomplishing technical goals that are not perceivable by humans or are not modeled in a human-centric fashion don't provide any added values as well.
For instance, measuring convergence by the distance of \ac{mfcc} vectors \citep[as done by][]{Han2018you} might provide an interesting technical view on the matter, but since humans don't converge simply by sounding more alike, this is, doubtfully an efficient user-friendly way to implement accommodation in \acp{sds}.
Viewing accommodation as an involved interdisciplinary research topic encourages collaboration of researchers studying linguistics, humanities subjects like sociology and psychology, conversation and user-experience designers, engineers, and anything between them.



% as a continuation to the technical challenges in modeling accommodation in computers
\todo{general sentence that there have been attempts to develop SDSes with accomm capabilities (with 1-2 refs) and say that those show different degrees of fidelity.}
Although all these designs are described as accommodative systems, not all accommodation capabilities are born equal.
This thesis distinguishes between several \enquote{levels of accommodation} in computers, as discussed in \cref{subsec:accommodation_levels}.
Ranging from the mere ability to modify the system's speech ability to independently generating varying realizations of change, but also allows for customizable conversational design complexity depending on the target application.
This concept is motivated by the parallelism to the assorted layers of accommodative behaviors in humans.
For example, in normal, everyday conversation, people speak spontaneously, and therefore their speech will change freely based on their personality, personal preference etc.
This means that no specific behavior is consciously targeted here and the changes will be arbitrarily varied around this general behavior.
In computers, this is the paralleled to the variational generation around a \enquote{base} behavior of the system shown in \cref{chap:statistical_model}, which, in turn, is extracted from different human productions.
However, in other, more controlled situations, different accommodation strategies might be more effective.
Teachers use entrainment as a means for giving auditory feedback to language learners, by triggering an artificially strong effect to \enquote{draw} the student into a more correct articulation form, e.g., of a specific sound or intonation pattern.
Although it is providing mostly implicit, this kind of feedback encourages learning from fluent, conversational responses.
Still, this requires a different, more directed approach.
Such approach is realized in this work via the pipeline presented in \cref{chap:computational_model}, which offers deterministic control over the system's responsiveness using several cognition-oriented parameters.
Such differentiation between accommodative mindsets has not been addressed before and it suggests more ways to model and implement accommodation in \acp{sds} while keeping the specific goal and application in mind, e.g., chatbots with a free-form accommodation process in contrast to \acp{capt} systems with a more well-defined goal.
















% toward the end, when talking about potential future research directions
Since computers are yet to possess full, human-level accommodation capabilities, it remains to be seen whether and how they will influence end users once they do.
First, like in the case of other human-inspired features like high-quality text generation and speech output, not all users might fancy such a capability that makes computers behave and perform more similar to humans.
A main reason for that the realistic yet imperfect attempt to adopt human behaviors often leads to the \emph{uncanny valley} effect \citep{Mori1970uncanny} and at some point makes users eerily uncomfortable \citep[cf.\ Figure 1 in][]{Macdorman2006subjective}.
Secondly, as in \ac{hhi}, some speakers are naturally less sensitive to phonetic changes and might not notice such variations in computers.
While accommodation effect might still occur in that case, this rises the questions whether this would improve user experience nonetheless and whether developers would want to invest in features that users might not even acknowledge and appreciate.
Finally, even when computers will have reached advanced accommodation capabilities (vocal and otherwise), they might not be accepted by users.
Depending on the application and the agent type, people might not \emph{want} their computers to demonstrate such human-like behaviors, especially if they don't necessarily explicitly follow the user's preference.
However, they might be useful and desirable in situations where the agent is designed to socially accompany a person for a long time.
For instance, assistant social robots or therapeutic virtual humans that can simulate realistic \ac{hhi} may achieve better rapport with their users, as the target is a closer long-term social relationship rather than the completion of isolated mundane tasks.
Somewhat ironically, this could only be comprehensively tested once such systems exist.

\todo[inline]{finish with a general sentence ending with the word "future"}