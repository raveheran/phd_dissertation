\chapter{Introduction}
\label{chap:introduction}

\lettrine{I}{ntroduction} chapter -- motivation and goals plus outline

\pagebreak

\section{Motivation and goals}
\label{sec:motivation_and_goals}

\section{Thesis outline}
\label{sec:outline}

This work deals with the intersection between two communicative phenomena, viz.\ \emph{phonetic accommodation} and \emph{\acl{hci}}.
Both of these topics play a role when talking with any kind of \emph{\acl{sds}}.
The challenges in combining them stem from the complexity and variability of accommodation processes and the absence of such naturally inherent human capability in computers.

\Cref{part:background} introduces the background and topics necessary for addressing this intertwinement.
\Cref{chap:phonetic_convergence} provides an overview of the theoretical, social, and linguistic aspects of accommodation in general, and particularly in spoken language.
This includes types of mutual variation throughout a conversation, measuring methods, and accommodation in \acl{hci}.
A survey of the ways humans interact with machines is presented in \cref{chap:human-computer_interaction}.
The properties and challenges of verbal interaction with computers are discussed as well.
\Cref{chap:spoken_dialogue_systems} gives an introduction to \aclp{sds}, along with the typical architecture of these systems and common ways to evaluate them.
Explanations about some systems in use nowadays and the differences between them are provided.
Finally, a roadmap for integrating accommodation capabilities into \aclp{sds} is introduced, as well as suggested terminology for differentiating some levels of accommodation in computers.

The main contributions are divided into three parts: Experiments, speech manipulation, and application.

A series of empirical convergence experiments are summarized in \cref{part:experiments}, each in a different social context and constellation of interlocutors.
\Cref{chap:conv_analysis} shows vocal accommodation effects and their utilization in real-world \emph{\acl{hhi}}.
Examining these effects in such conversations helps to determine the gaps between the analysis of conversations in the wild and lab setting.
Due to the length of these conversations, analyses of both dynamic changes over time and more general classification of speaker type are possible.
\Cref{chap:shadowing_in_sung_music_and_human_computer_interaction} presents shadowing tasks combining both \emph{\acl{hhi}} and \emph{\acl{hci}} contexts.
These tasks were carried out in closely controlled experimental settings for direct comparison between the two contexts.
Lastly, a multiparty, \emph{\acl{hhci}} experiment is outlined in \cref{chap:speech_variations_in_hhci}.
This more evolved mix of speakers sheds light on accommodation effects influenced by the addressee of the specific utterance or by the presence of another human interlocutor.

\Cref{part:speech_manipulation} comprises techniques for on-demand, real-time manipulation of synthetic speech, which would enable the required control over a system's output in order to support accommodation capabilities.
\todo[inline]{here details about the chapters in this part once they exist}

\Cref{part:application} contains implementations of steps on the way of achieving a responsive \acl{sds}.
First, approaches for modeling accommodation are presented in \cref{chap:computational_model,chap:statistical_model}.
These include a computational approach based on empiric data and a statistical approach using time series.
Then, the technical details of a module linking between the speech input and output of a system are described in \cref{chap:convergence_module_for_sdss}.
Together with the modeling information and the techniques from \cref{part:speech_manipulation}, this module grants accommodation capabilities to its \acl{sds}.
Finally, an \acl{e2e} web-based system is introduced in \cref{chap:web-based_responsive_spoken_dialogue_system}.
The extended architecture, usage options, and  graphical visualizations are demonstrated via a use-case display.

\label{outline_end}

\todo[inline]{any additional remarks to mention in another section?}