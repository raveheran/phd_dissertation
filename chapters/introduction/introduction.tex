\chapter{Introduction}
\label{chap:introduction}

\lettrine{S}{poken} human communication is complex and dynamic.
One reason for that is the influence interlocutors have on each other while talking.
The rise of voice-activated devices challenges the way people use their voice to communicate.
Since computers do not poses the same inert tendency to be influenced by human speech, it is an open question whether human communicate the same with computers and whether computers can simulate these natural dynamic changes.
This chapter provides the motivation for studying verbal \acl{hci} and the main goals of this work.

\pagebreak

\acresetall

\section{Motivation and goals}
\label{sec:motivation_and_goals}

People tend to adopt certain behavioral patterns from one another while interacting.
These may range from simple physical postures to language usage and even emotional reactions.
The overarching term for this phenomenon is \emph{accommodation} and it is commonly occurring in \acl{hhi}.
As explained by \acl{cat} from the 1970s, accommodation often has a social motive and it is used, even if unconsciously, to identify oneself with certain addressees or to trigger greater likability among a social group \citep{Giles2007CAT}.
It is theorized that the intrinsic motivation for this mechanism is a decrease (or an increase) of the social distance between interlocutors and the improvement of the interaction's overall efficiency \citep{Gallois2015CAT}.
Various empirical experiments have found accommodation effects in a variety of modalities, like facial expressions \citep{Kinsbourne2009embodied}, eye gaze \citep{Leong2017speaker}, lexical choices \citep{Brennan1996lexical}, and more.
Changes in speech, and especially low-level phonetic ones, are sometimes more subtle and harder to spot, e.g., than a mimic of body posture or the repeated use of a specific word.
Nevertheless, accommodation effects have been found in speech-related features, like speech rate \citep{Levitan2011measuring, Local2007phonetic} and pitch contour \citep{Babel2012role}.

The automatic utilization of accommodation strategies by speakers makes it an integral part of human communication.
However, in recent years, the everyday usage of voice-activated devices has been consistently increasing.
This kind of interaction introduces new challenges and coerces humans to adjust their verbal communication to cope with the limited capabilities of computer-based interlocutors.
While similar accommodation effects have also been found in \acl{hci} in different experimental settings \citep[e.g.,][]{Bell2003prosodic, Levitan2013entrainment, Parent2010lexical}, the effects were present mostly on the human side, whereas the systems' accommodation capabilities were considerably more limited at best.
Such one-sided adaptation is incongruous with the mutual, dynamic exchanges occurring in \acl{hhi}.
Expanding the effect to computer interlocutors to simulate the aforementioned conversational dynamics still poses a challenge.
This challenge comprises both technical and modeling facets.
The former deals with the ability of a system to detect phonetic changes in the human's speech and to manipulate the corresponding features in its output speech in real-time, while the latter refers to determining the relationship between a user's realizations and the way they influence the system's productions.
Direct control over synthesized speech is challenging due to the limitations of current \acl{tts} synthesis methods, which almost exclusively use a trained model that cannot be modified on-the-fly.
Even if such capabilities are achieved, accommodation models are required for establishing the way the system responds to users' speech variations.
This involves some design decisions.
For example, should the system try to simulate behaviors observed in \aclp{hhi} or simply follow the user's lead?
Should the system initiate changes or only react to user variation?
Could the course of the interaction be influenced by the way the system adapts towards the user?
All these decisions are part of defining the dynamics between the human and computer interlocutors and may change depending on the specific application.

Integration of accommodation capabilities can be especially beneficial for \aclp{sds}, as they are typically the core of verbal \acl{hci}.
Like improvements in other aspects of \aclp{sds}, vocal accommodation capabilities will contribute to their enhancement toward more human-like conversational behavior \citep{Weise2017towards}.
Considering the assumptions of \acl{cat}, interacting with a system that simulates behaviors familiar from \acl{hhi} should ease the process for the user and ultimately make it more efficient and fluent.
Furthermore, with the usage growth of voice-activated devices like \aclp{pa}, accommodative speech would offer additional degrees of personalization for users.
For instance, the speech of a \acl{pa} owned by one user might differ from that of another and could change when encountering a new user -- therefore reflecting the adjustments performed by humans.
Furthermore, studying accommodation in \acl{hci} shed light on the way humans perceive non-human interlocutors in social contexts and whether they want to communicate with them in a similar manner as with other humans.
\citet{Benus2018prosodic} show that a computer-based interlocutor gained more trust from human companions when it exhibited some level of vocal accommodation.

This work investigates the building blocks on the way to achieving vocal accommodation in \acl{hci}.
These include experiments for collecting evidence of accommodative behaviors in \acl{hhi} and \acl{hci}, approaches for modeling these behaviors in a computer-compatible fashion, methods for integrating accommodation models into real-time \acl{tts} synthesis, and implementation of a \acl{sds} that support vocal accommodation.
Previous work has addressed these concepts, mostly independently of each other.
\citet{Levitan2016implementing} introduce an approach for integrating prosodic-acoustic convergence into a conversational avatar, but without considering different types of accommodative behaviors.
Similarly, \citet{Bevnuvs2014social} examines social aspects of entrainment in spoken interactions, but does not demonstrate how those can be harnessed to measure them and develop models.
Obviously, the scope of each study cannot possibly cover all topics.
However, in addition to the depth of each of these concepts, the connections between them for introducing a complete solution should be considered as well.
For example, the manner in which the experimental findings are converted into a model defines the flexibility and degree of variation of the system.
It is therefore important to jointly address both the theoretical and technical facets of the topic, as they can benefit each other.
On the one hand, the technical capability to manipulate speech needs a modeled knowledge about the possible (and plausible) changes that might occur; and on the other hand, accumulating empirical data without showing how it models the phenomenon in question makes it highly challenging to demonstrate the essence of the captured evidence.

Offering such a comprehensive overview of this multidisciplinary theme and presenting the individual topics in a wider context were the primary inspirations for this work.
A further motivation was to suggest a more structural approach to accommodation description in computers, namely a hierarchy of accommodation levels.
Each level builds on the previous one and progressively increases the complexity and variability of the accommodative behavior, from direct mirroring of users' productions to independent responses
To that end, empirical data is required for observing a range of behaviors, and appropriate computational means need to be utilized to prevent too simple or unnecessarily complex behaviors.
This distinguishment between different types of behavior has received little to no attention so far and can help to better define the desired behavior of a system, based on user's expectation and the target application.
Lastly, an emphasis is put on the temporal aspect of conversation -- and by extension, of accommodation effects -- throughout the work, which is often neglected in studies, but provides important insights on the interactions' dynamics.

\section{Outline}
\label{outline} % intentionally doesn't have prefix sec: 

This work lies at the intersection of two communication phenomena, viz.\ \emph{phonetic accommodation} and \emph{\acl{hci}}.
Both of these topics play a role when talking with any kind of \emph{\acl{sds}}.
The challenges in combining them stem from the complexity and variability of accommodation processes and the absence of this inherent human capability in computers.
The core parts are structured to reflect the journey from theoretical ideas and empirical experiments, through modeling, and ultimately implementation of potential applications.

\Cref{part:background} introduces the main topics related to this intertwinement of research areas.
\Cref{chap:phonetic_convergence} provides an overview of the theoretical, social, and linguistic aspects of accommodation in general, and in spoken language in particular.
This includes types of mutual variation throughout a conversation and measurement of accommodation in \acl{hci}.
A survey of the ways humans interact with machines is presented in \cref{chap:spoken_dialogue_systems}.
The properties and challenges of verbal interaction with computers are discussed as well.
This chapter also introduces \aclp{sds}, along with their typical architecture and examples of common modern applications of them.
Finally, a suggested roadmap for integrating accommodation capabilities into \aclp{sds} is explained, together with terminology for differentiating some levels of accommodation in computers.

The main contributions are subsequently divided into three parts: Experiments, Modeling, and Application.
A series of empirical accommodation experiments are described in \cref{part:experiments}, each in a different social context and constellation of interlocutors.
\Cref{chap:conv_analysis} shows vocal accommodation effects and their utilization in real-world \emph{\acl{hhi}}.
Examining these effects in such conversations helps to determine the gaps between the analysis of conversations in the wild and lab setting.
Due to the length of these conversations, analyses of both dynamic changes over time and more general classification of speaker type are possible.
\Cref{chap:shadowing_in_sung_music_and_human_computer_interaction} presents shadowing tasks combining both \emph{\acl{hhi}} and \emph{\acl{hci}} contexts.
These tasks were carried out in closely controlled experimental settings for direct comparison between the two contexts.
Further evidence of accommodation in a different context is explored in a study of vocal accommodation in \emph{singing}.
Lastly, a multiparty \emph{\acl{hhci}} study is outlined in \cref{chap:speech_variations_in_hhci}.
This more evolved mix of speakers sheds light on accommodation effects influenced by the addressee of the specific utterance or by the presence of another human interlocutor.

\Cref{part:modeling} comprises two approaches for modeling accommodation to be used in a \acl{sds}.
A computational model composed of empirically-motivated parameters is introduced in \cref{chap:computational_model}.
This model aims to provide a descriptive way to depict accommodation and craft desired behaviors.
The approach taken in \cref{chap:statistical_model} is statistical in nature.
It identifies different speaker tendencies, even if those cannot be explicitly be broken into specific properties.
Nevertheless, these can be used for defining a target behavior for a system.
The use of these two approaches in conjunction is also discussed.

\Cref{part:application} contains implementations of components required for responsive \aclp{sds}.
The technical details of a module linking between the speech input and output of a system are described in \cref{chap:convergence_module_for_sdss}.
This includes an additional module for handling accommodation-related processes, like detecting variable sounds and running the model, and a brief survey of real-time, on-demand manipulation of synthesized speech, which enables the required control over a system's output needed for realizing accommodation effects.
Together with the modeling information and the techniques from \cref{part:modeling}, these components are utilized in the system introduced in \cref{chap:web-based_responsive_spoken_dialogue_system}.
The extended architecture, usage possibilities, and graphical visualizations of this system are demonstrated via a use-case display.

\label{outline_end}
\vfill

\pagebreak

\section{Related own publications}
\label{sec:related_own_publications}

\begin{longtable}{m{.35\linewidth}m{.6\linewidth}}
	\toprule
	{\large \bfseries \Cref{chap:shadowing_in_sung_music_and_human_computer_interaction} --\newline\newline \nameref{chap:shadowing_in_sung_music_and_human_computer_interaction}} &
	\fullcite{Raveh2020SpeechProsody}\newline\newline
	\fullcite{Raveh2017ESSV}\\
	%
	\midrule
	%
	{\large \bfseries \Cref{chap:speech_variations_in_hhci} --\newline\newline \nameref{chap:speech_variations_in_hhci}} &
	\fullcite{Raveh2019InterspeechAlexa}\newline\newline
	\fullcite{Raveh2019ESSV}\\
	%
	\midrule
	%
	{\large \bfseries \Cref{chap:computational_model} --\newline\newline \nameref{chap:computational_model}} &
	\fullcite{Raveh2017Interspeech}\newline\newline
	\fullcite{Raveh2019PundP}\\
	%
	\midrule
	%
	{\large \bfseries \Cref{chap:convergence_module_for_sdss} --\newline\newline \nameref{chap:convergence_module_for_sdss}} &
	\fullcite{Raveh2017SemDial}\newline\newline
	\fullcite{Raveh2017PundP}\\
	%
	\midrule
	%
	{\large \bfseries \Cref{chap:web-based_responsive_spoken_dialogue_system} --\newline\newline \nameref{chap:web-based_responsive_spoken_dialogue_system}} &
	\fullcite{Raveh2018Specom}\newline\newline
	\fullcite{Raveh2021ICASSP}\\
	\bottomrule	
\end{longtable}