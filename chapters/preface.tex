\pagenumbering{Roman} % use roman page numbering for preface

% *** Dedication ***
% ==================
\vspace*{9cm}
\begin{center}
%        \textit{dedication text here}
\end{center}
\newpage

% *** Acknowledgments ***
% =======================
\vspace*{3cm}
\begin{center}
    {\textbf{\huge Acknowledgments}}\\[0.3cm]
    \Large
    % Acknowledgments and thanks here (names in bold)
\end{center}
\newpage

% *** Forword ***
% ===============
{\calligra \Huge \textit{Forword}}
\clearpage        

% *** Disclaimer ***
% ==================
\begin{multicols}{2}
    \vspace*{\textheight}
    \columnbreak
    \vspace*{3cm}
    \Large{\textit{Hiermit versichere ich, dass ich die vorgelegte Arbeit selbstst{\"a}ndig und nur mit den angegebenen Quellen und Hilfsmitteln einschließlich des WWW und anderer elektronischer Quellen angefertigt habe.
            Alle Stellen der Arbeit, die ich anderen Werken dem Wortlaut oder dem Sinne nach entnommen habe, sind kenntlich gemacht.}}
    \vspace{0.55cm}
    \begin{flushright}
%       {\stonefont Eran Raveh}
		{Eran Raveh}
    \end{flushright}
\end{multicols}
\cleardoublepage

\pagestyle{fancyplain}
\fancyhead{} % no header (for now)
\renewcommand{\headrulewidth}{0pt} % supress header line
\cfoot{} % reset current footer (e.g., default page number)
\fancyfoot[LO,RE]{\thepage} % set footer on all pages

% *** Abstract ***
% ================
\begin{center}
    \vspace*{3.3cm}
    \textit{\textbf{\huge Abstract}}\\[0.3cm]
    \Large
    English text
\end{center}
\clearpage

% *** Abstrakt (Deutsch) ***
% ================
\begin{center}
     \vspace*{3.3cm}
     \textit{\textbf{\huge Zusammenfassung (Deutsch)}}\\[0.3cm]
     \Large
	 \foreignlanguage{german}{
	 	Deutscher Text
	 }
\end{center}
\cleardoublepage

% table of contents
%------------------
\renewcommand{\contentsname}{\hfill\bfseries\Huge Table of Contents\hfill}
\renewcommand{\cftaftertoctitle}{\hfill}
\setlength\cftaftertoctitleskip{60pt} % space between TOC and its title
%\thispagestyle{empty}
% put the word ``page'' above page number column
{\addtocontents{toc}{~\hfill\textbf{Page}\par}
    {\large{} \tableofcontents \normalsize{}}}

% list of figures
\newpage
\addcontentsline{toc}{chapter}{\listfigurename}
\listoffigures

% list of tables
\newpage
\addcontentsline{toc}{chapter}{\listtablename}
\listoftables

% list of formulae
\newpage
\addcontentsline{toc}{chapter}{List of Formulae and Algorithms}
\listofmyequations

% list of acronyms
\newpage
\section*{List of Acronyms}
\addcontentsline{toc}{chapter}{List of Acronyms}
\vspace*{0.2cm}
\begin{multicols}{2}
	\printacronyms
\end{multicols}
\clearpage

% notation
\section*{Notation}
\addcontentsline{toc}{chapter}{Notation}
\begin{tabularx}{\linewidth}{l@{\quad}X}
	$\vec{r}$ & vector $r$ \\
	$\mathcal{A}^{m \times n}$ & matrix $A$ with dimension $m$ over $n$ \\
	$\mathcal{A}^{\mathbb{R} \times \mathbb{R}}$ & matrix $A$ describing a 2-dimensional space of real numbers \\
	$\mathcal{G}: \mathbb{Q}^{n \times m} \longrightarrow \mathbb{Q}^{m}$ & a function $\mathcal{G}$ that maps a $n \times m$ rational numbers matrix to a $m$-dimensional vector of rational numbers.\\
	$\mathcal{N}$ 	&	normal (Gaussian) distribution \\
	$X \sim \mathcal{N}(\mu,\,\sigma^{2})$	&	random variable $X$ has a normal distribution with mean $\mu$ and variance $\sigma^{2}$ \\
	$k\left( \theta, x, x' \right)$ & covariance function (kernel) $k$ between all possible input pairs using hyperparameter vector $\theta$. Can be shorted-noted as $\mathcal{K}$. $x$ and $x'$ are two data vectors. \\
	$\Sigma(\vec{x})$	&	covariance matrix (e.g., of a Gaussian process). Given by $\Sigma_{i,j} = k(x_i, x_j)$, where k is a positive definite kernel function \\
	$p\left( f \left( \vec{x} \right) \right) \sim \mathcal{GP}\left( m(\vec{x}), \Sigma(\vec{x}) \right)$	&	probability of values of function $f$ over vector $x$ is specified by a Gaussian process with mean function $m$ and covariance function $K$ \\
	$f(X)$	&	a vector of function values, whose $i$th element is given by $f(x_i)$ \\
	$x_*$	&	a not-yet-observed outcome value. Similarly, $f_*$ collectively denotes all non-observed outcome values of a function, and $\Sigma_*$ the covariance values of non-observed values \\
	$\lVert \mathbf{p - q} \rVert_d$ & the Euclidean distance between points $p$ and $q$ in a $d$-dimensional space\\
	$\vcenter{\hbox{\includegraphics[height=15pt]{qt_c-cih}}}$	& a note raised by one \acl{qt}\\
	$\vcenter{\hbox{\includegraphics[height=15pt]{qt_c-cisih}}}$	& a note raised by three \aclp{qt}\\
\end{tabularx}
  	
% outline
\newpage
\section*{Outline}
\addcontentsline{toc}{chapter}{Outline}
%
This work deals with the intersection between two communicative phenomena, viz.\ \emph{phonetic accommodation} and \emph{\acl{hci}}.
Both of these topics play a role when talking with any kind of a \acl{sds}.
\Cref{part:introduction} introduces the background necessary for addressing this intertwinement.
\Cref{chap:phonetic_convergence} provides an overview of the theoretical, social, and linguistic aspects of accommodation in general, and particularly in spoken language.
This includes types of mutual variation throughout a conversation, including examples of short- and long-term changes.
A survey of the ways humans interact with machines is presented in \cref{chap:human-computer_interaction}, where properties and challenges of verbal interaction with computers are discussed as well.
\Cref{chap:spoken_dialogue_systems} gives an introduction to \aclp{sds}, along with explanations about systems in use nowadays and the differences between them.
The typical architecture of such systems and common ways to evaluate them are described as well.

The main contributions are divided into three parts: Experiment, speech manipulation, and application.
A series of empirical convergence experiments are summarized in \cref{part:experiments}, each in a different social context and constellation of interlocutors.
\Cref{chap:conv_analysis} shows vocal accommodation effects and their utilization in real-world \emph{\acl{hhi}}.
Examining these effect in such conversations  allows determining the gaps between conversation in the wild and in labs.
Due to the length of these conversations, analyses of both dynamic changes over time and more general classification of speaker type are possible.
\Cref{chap:shadowing_experiment_with_natural_and_synthetic_voices} presents shadowing tasks combining both \emph{\acl{hhi}} and \emph{\acl{hci}} contexts.
These tasks were carried out in closely controlled experimental settings for direct comparison between the two contexts.
Lastly, a multiparty, \emph{\acl{hhci}} experiment is outlined in \cref{chap:speech_variations_in_hhci}.
This more evolve mix of speakers sheds light on accommodation effects influenced by the addressee of the specific utterance or by the presence of another human interlocutor.
\Cref{part:speech_manipulation} comprises techniques for on-demand, real-time manipulation of speech, which would enable the required control over a system's output in order to support accommodation capabilities.
\todo[inline]{here details about the chapters in this part once they exist}
\noindent\Cref{part:application} contains implementations of steps on the way of achieving a responsive \acl{sds}.
First, approaches for accommodation modeling are presenting in \cref{chap:computational_model,chap:statistical_model}.
Then, technical details of a module linking between the speech input and output of a system is introduced in \cref{chap:convergence_module_for_sdss}.
Together with the modeling information and the techniques from \cref{part:speech_manipulation}, this module grants accommodation capabilities to its \acl{sds}.
Finally, a web-based \acl{e2e} system is presented in \cref{chap:web-based_responsive_spoken_dialogue_system}.
The extended architecture, usage options, and  graphical visualizations are described via a demonstrated use-case of such a system.

% set pages header and footers (moved from preamble because only need headers from here on (and better be after all lists etc.))
%-----------------------------
\clearpage % start new layout from next page
\pagestyle{fancy} % reset layout
\renewcommand{\headrulewidth}{0.4pt} % revive header line
\renewcommand{\chaptermark}[1]{\markboth{Chapter~\thechapter~--~#1}{}} % customize chapter header
\renewcommand{\sectionmark}[1]{\markright{\thesection\quad#1}} % customize subsection header
%\fancyhead{} % reset header
\fancyhead[LO]{\leftmark} % set right header on odd pages
\fancyhead[RE]{\rightmark} % set leftheader on odd pages

\addtocounter{page}{1} % reset page counter after preface
\pagenumbering{arabic} % switch to arabic page numbers