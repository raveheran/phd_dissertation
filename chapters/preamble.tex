% ==========================================
% 				PACKAGES
% ==========================================

% *** text packages ***
% =====================
\usepackage[utf8]{inputenc} % for direct UTF-8 text
\usepackage[T1]{fontenc} % font encoding
\usepackage{fontspec} % control over fonts
\usepackage[hyphens]{url} % URL handling (must be before hyperref)
\usepackage[hidelinks, pdfusetitle, hyperfootnotes=false]{hyperref} % for hyper references
%\usepackage[anythingbreaks]{breakurl} % for line-wrapping URLs
\usepackage{fullpage}
\usepackage{multicol} % enable multi-columns text
\usepackage{multirow} % enable multi-rows text
%\usepackage[style=numeric,backend=biber]{biblatex}
\usepackage{lettrine} % for using drop-down letters
%\usepackage{draftwatermark} % put ``draft'' water mark on all pages
\usepackage[inline, shortlabels]{enumitem}
%\usepackage{enumerate} % for using enumerations
\usepackage{acro} % for acronyms
\usepackage[detect-all]{siunitx} % SI units, numbers
\sisetup{list-final-separator={, and }}
\usepackage{tipa} % enable usage of IPA symbols
%\usepackage{cite}
\usepackage[ruled, linesnumbered]{algorithm2e} % for algorithms
\usepackage{algpseudocode} % enable usage of psudo-code blocks
\usepackage{pifont} % for more symbols
\usepackage{amsmath} % for extended mathematical expressions
\usepackage{amssymb} % for more math symbols
\usepackage{lingmacros} % for ``linguistic'' listing
\usepackage{fancyhdr} % for more control over header/footer desiging
\usepackage{sectsty} % more options for sections
\usepackage{etoolbox} % more control over the bibliography
\usepackage{xcolor} % extended colors set (table is for colors in tables)
\usepackage[capitalise, nameinlink, noabbrev]{cleveref} % cross-referencing
\usepackage{fancyvrb} % fancy verbatim

% *** L10n ***
% ------------
\usepackage[ngerman, english]{babel} % for better handling of text (e.g. line breaks)
\usepackage[autostyle]{csquotes} % quoating style

% *** graphic, layouts and visualization packages ***
% --------------------------------------------------
\usepackage[subfigure]{tocloft} % for more control over table of contents
\usepackage{listings} % for code highlighting
\usepackage{setspace} % for more control over spacing configurations
\usepackage{graphicx} % extended graphic capabilities
\usepackage{float} % more control over floats
\usepackage{eso-pic} % for some special images layouts (e.g., for cover page)
\usepackage[cc]{titlepic} % for putting picture in cover page
\usepackage{subfigure} % subfigures
%\usepackage[singlelinecheck=off, justification=raggedright]{subcaption} % multiple captions
%\captionsetup{compatibility=false}
\usepackage{adjustbox}
\usepackage[graphicx]{realboxes}
\usepackage{rotating}
\usepackage{pdflscape} % for rotating pages (also in the pdf itself)
\usepackage[headsep=1cm,headheight=2cm]{geometry} % for changing page margins
\usepackage{hhline} % for more evolved designing of tables
\usepackage{wrapfig} % for wrapped figures and tables
\usepackage{qtree} % for syntactic trees
\usepackage{caption}
% \usepackage{tikz}
% \usetikzlibrary{decorations.pathmorphing}
% \usetikzlibrary{decorations.markings}
% \usetikzlibrary{arrows}

% *** table layout ***
% --------------------
\usepackage{tabularx}
\usepackage{tabulary}
\usepackage{booktabs} % more control over tables design
\usepackage{longtable} % for making multi-page tables
\usepackage{makecell}

% *** additional packages ***
% --------------------------------------------------
\usepackage{titlesec} % for controlling title formatting
\usepackage{lmodern} % for avoiding font-shape and letter-size warnings
\usepackage[colorinlistoftodos]{todonotes} % for todo notes on PDF
%\usepackage[round]{natbib} % references design
%\usepackage{apacite} % APA-style citation, automatically sorted alphabetically

% ==========================================
% 				CONFIGURATIONS
% ==========================================

% font definitions
% ----------------
% (font information and examples on http://www.tug.dk/FontCatalogue/allfonts.html)
\usepackage{calligra} % caligraphy font (to apply, use \calligra)
\input AnnSton.fd
\newcommand*\stonefont{\usefont{U}{AnnSton}{xl}{n}} % use \stonefone for stone-like capital blocks
\usepackage{emerald}
\renewcommand\theadfont{\bfseries}

%\DeclareMathSizes{30}{20}{10}{8} % set font sizes of math environments

% \SetWatermarkScale{5.5} % scale the size of the ``draft'' watermark

\linespread{1.3} % line spreading in file
\setlength{\footnotesep}{0.5cm} % space between footnotes
\setlength{\skip\footins}{1cm} % space between text and footnotes
\urlstyle{same} % text style of url links

\crefformat{footnote}{#2\footnotemark[#1]#3} % format of footnote references
\creflabelformat{equation}{#2\textup{#1}#3} % don't put equation references in parentheses

\graphicspath{{figures/}} % global path for figures

% configuring list of equations for table of contents
\newcommand{\listequationsname}{List of Formulae and Algorithms}
\newlistof{myequations}{equ}{\listequationsname}
\newcommand{\eqname}[1]{\addcontentsline{equ}{myequations}{\protect\numberline{\theequation}#1}\par} % add equation to ToC
\setlength{\cftmyequationsnumwidth}{2.5em} % width of equation number in List of Equations

\DeclareMathOperator*{\argmin}{\mathit{argmin}}
\DeclareMathOperator*{\argmax}{\mathit{argmax}}

% rotation configurations for tables
% ---------------------------------------
%\newcommand*\rot{\rotatebox{90}}
%\newcommand{\mcrot}[4]{\multicolumn{#1}{#2}{\rlap{\rotatebox{#3}{#4}~}}}

\numberwithin{equation}{section} % include section in equation numbering
\setcounter{secnumdepth}{5} % setting level of numbering
\setcounter{tocdepth}{2} % setting level of numbering for table of contents

% PDF file settings
% -----------------
\hypersetup {
    pdftitle={Eran Raveh -- Ph.D. Dissertation},
    pdfauthor={Eran Raveh},
    pdfsubject={Ph.D. Dissertation},
    pdfkeywords={PhD dissertation, human-computer interaction, speech synthesis, spoken dialog systems, phonetic convergence},
    bookmarksnumbered=true,
    bookmarksopen=true,
    bookmarksopenlevel=1,
%    breaklinks=true,
    colorlinks=false, % ** change to false for hardcopy printing **
    citecolor=blue, % the color of cite links
    urlcolor=blue, % color of url links
    linkcolor=blue, % color of other links
    pdfpagemode=UseOutlines, % open with no panes
    pdfpagemode=UseNone % in case we want no panes to be opened
    pdfpagelayout=TwoPageRight % 2-pages view
    pdfstartview=FitH,  % fit page width
}

% define additional colors
% ------------------------
\definecolor{maroon}{rgb}{0.5,0,0}
\definecolor{darkgreen}{rgb}{0,0.5,0}

% configurations of algorithm blocks
% ----------------------------------
\newcommand\mycommfont[1]{\footnotesize\ttfamily\textcolor{gray}{#1}}
\SetCommentSty{mycommfont}
\lstset {
	basicstyle=\ttfamily,
	numberstyle=\footnotesize,
	numbers=left,
	stepnumber=10,
	xleftmargin=1.5em,
	xrightmargin=1.5em,
	%	backgroundcolor=\color{gray!10},
	tabsize=2,
	rulecolor=\color{black!30},
	breaklines=true,
	breakatwhitespace=true,
	framextopmargin=2pt,
	framexbottommargin=2pt,
	inputencoding=utf8,
	literate={ä}{{\"a}}1 {ü}{{\"u}}1
}
\makeatletter
\newcommand{\algorithmcaption}[2][\footnotesize]{%
	\let\old@algocf@finish\@algocf@finish% Store algorithm finish macro
	\def\@algocf@finish{\old@algocf@finish% Update finish macro to insert "footnote"
		\leavevmode\rlap{\begin{minipage}{\linewidth}
				#1#2
		\end{minipage}}%
	}%
}
\makeatother

% configurations of XML blocks
% ----------------------------
\lstdefinelanguage{XML} {
    basicstyle=\ttfamily,
    morestring=[s]{"}{"},
    morecomment=[s]{?}{?},
    morecomment=[s]{!--}{--},
    commentstyle=\color{maroon},
    moredelim=[s][\color{black}]{>}{<},
    moredelim=[s][\color{blue}]{\ }{=},
    stringstyle=\color{red},
    identifierstyle=\color{darkgreen}
}

% set pages header and footers
%-----------------------------
\pagestyle{fancy}
%\renewcommand{\sectionmark}[1]{\markright{#1}{}}
%\fancyhf{}
%\lhead{\fancyplain{}{\rightmark}} % 1. sectionname, 1.1 subsection name etc
\fancyhead{}
\fancyhead[LO]{\leftmark}
\fancyhead[RE]{\rightmark}

%\fancyfoot{}
%\fancyhead[RO,LE]{\thepage}
%\fancyhead[LO]{\leftmark}
%\fancyhead[RE]{\rightmark}
%\fancyhead[RE]{\section}
%\fancyfoot[CE,CO]{\leftmark}
%\fancyfoot[LE,RO]{\thepage}

% todo items commands
% -------------------
%\renewcommand{\todo}[1]{\todo[fancyline]##1}
\newcommand{\review}[1]{\todo[color=green]{#1}}
\newcommand{\fixme}[1]{\todo[color=red, fancyline]{#1}}
\newcommand{\putref}[1]{\todo[color=blue!40]{#1}}

% acronym managment
% -----------------
\acsetup{only-used=false, list-name=List of Acronyms, page-style=plain, pages=all, list-style=description, list-heading=none} % list-heading=section*
\DeclareAcronym{ae}		{short=AE,		long=account executive,								long-plural=s, short-plural=s}
\DeclareAcronym{ai}		{short=AI,		long=artificial intelligence,						long-plural=s, short-plural=s}
\DeclareAcronym{ar}		{short=AR,		long=articulation rate,								long-plural=s, short-plural=s}
\DeclareAcronym{asp}	{short=ASP,		long=additional speech processing,  				long-plural=, short-plural=}
\DeclareAcronym{asr}	{short=ASR, 	long=automatic speech recognition, 					long-plural=, short-plural=}
\DeclareAcronym{b2b}	{short=B2B, 	long=business-to-business,							long-plural=, short-plural=}
\DeclareAcronym{cat}	{short=CAT, 	long=communication accommodation theory, 			long-plural=, short-plural=}
\DeclareAcronym{call}	{short=CALL, 	long=computer-assisted language learning, 			long-plural=, short-plural=}
\DeclareAcronym{capt}	{short=CAPT, 	long=computer-assisted pronunciation training, 		long-plural=, short-plural=}
\DeclareAcronym{c-ai}	{short=C-AI, 	long=conversational \acs*{ai},						long-plural=s, short-plural=s, list=conversational \acs*{ai}}
\DeclareAcronym{ci}		{short=CI, 		long=conversation intelligence, 					long-plural=, short-plural=, alt=C-IQ, extra=a.k.a.\ conversation IQ}
\DeclareAcronym{cmc}	{short=CMC, 	long=computer-mediated communication, 				long-plural=s, short-plural=s}
\DeclareAcronym{cnc}	{short=C\&C, 	long=command and control, 							long-plural=, short-plural=}
\DeclareAcronym{cc}		{short=CC, 		long=cross-correlation,								long-plural=s, short-plural=s}
\DeclareAcronym{crm}	{short=CRM, 	long=customer relations management,					long-plural=, short-plural=s}
\DeclareAcronym{crqa}	{short=CRQA, 	long=cross-recurrence quantification analysis,		long-plural-form=cross-recurrence quantification analyses, short-plural=s}
\DeclareAcronym{CRQA}	{short=CRQA, 	long=cross-recurrence quantification analysis,		long-plural-form=cross-recurrence quantification analyses, short-plural=s}
\DeclareAcronym{dds}	{short=DDS, 	long=device-directed speech, 						long-plural=es, short-plural=es}
\DeclareAcronym{dl}		{short=DL,		long=deep learning,									long-plural=, short-plural=}
\DeclareAcronym{dm}		{short=DM,		long=dialogue manager,								long-plural=s, short-plural=s}
\DeclareAcronym{e2e}	{short=E2E,		long=end-to-end,									long-plural=, short-plural=}
\DeclareAcronym{f0}		{short=f$_0$,	long=fundamental frequency, 						long-plural=, short-plural=}
\DeclareAcronym{gp}		{short=GP,		long=Gaussian process,								long-plural=es, short-plural=es}
\DeclareAcronym{gui}	{short=GUI, 	long=graphical user interface, 						long-plural=s, short-plural=s}
\DeclareAcronym{hds}	{short=HDS, 	long=human-directed speech, 						long-plural=es, short-plural=es}
\DeclareAcronym{hci}	{short=HCI, 	long=human-computer interaction,					long-plural=s, short-plural=s}
\DeclareAcronym{hhi}	{short=HHI, 	long=human-human interaction,						long-plural=s, short-plural=s}
\DeclareAcronym{hhci}	{short=HHCI, 	long=human-\acl*{hci},								long-plural=s, short-plural=s, list=human-\acl*{hci}}
\DeclareAcronym{hmm}	{short=HMM, 	long=hidden markov model, 							long-plural=s, short-plural=s}
\DeclareAcronym{ipa}	{short=IPA,		long=international phonetic alphabet, 				long-plural=, short-plural=}
\DeclareAcronym{its}	{short=ITS, 	long=intelligent tutoring system, 					long-plural=s, short-plural=s}
\DeclareAcronym{iqr}	{short=IQR, 	long=interquartile range,							long-plural=s, short-plural=s}
\DeclareAcronym{iva}	{short=IVA,		long=intelligent virtual agent, 					long-plural=s, short-plural=s}
\DeclareAcronym{loess}	{short=LOESS, 	long=locally estimated scatterplot smoothing, 		long-plural=, short-plural=}
\DeclareAcronym{los}	{short=LoS, 	long=line of synchrony,								long-plural-form=lines of synchrony, short-plural=s}
\DeclareAcronym{ltas}	{short=LTAS, 	long=long-term average spectrum,					long-plural=s, short-plural=es}
\DeclareAcronym{mfcc}	{short=MFCC, 	long=mel-frequency cepstral coefficient,			long-plural=s, short-plural=s}
\DeclareAcronym{ml}		{short=ML, 		long=machine learning, 								long-plural=, short-plural=}
\DeclareAcronym{ngs}	{short=NGS, 	long=non-goal system, 								long-plural=s, short-plural=s}
\DeclareAcronym{nlg}	{short=NLG, 	long=natural language generation, 					long-plural=, short-plural=}
\DeclareAcronym{nlp}	{short=NLP, 	long=natural language processing, 					long-plural=, short-plural=}
\DeclareAcronym{nlu}	{short=NLU, 	long=natural language understanding, 				long-plural=, short-plural=}
\DeclareAcronym{pa}		{short=PA, 		long=personal assistant, 							long-plural=s, short-plural=s}
\DeclareAcronym{rmse}	{short=RMSE, 	long=root-mean-square error,						long-plural=s, short-plural=s}
\DeclareAcronym{roi}	{short=ROI, 	long=return on investment,							long-plural-form=return on investments, short-plural=s}
\DeclareAcronym{rr}		{short=RR,	 	long=recurrence rate,								long-plural=s, short-plural=s}
\DeclareAcronym{rqa}	{short=RQA,	 	long=recurrence quantification analysis,			long-plural-form=recurrence quantification analyses, short-plural=s}
\DeclareAcronym{sd}		{short=sd, 		long=standard deviation, 							long-plural=s, short-plural=s}
\DeclareAcronym{sdr}	{short=SDR, 	long=sales development representative,				long-plural=s, short-plural=s}
\DeclareAcronym{sampa}	{short=SAMPA, 	long=speech assessment methods phonetic alphabet, 	long-plural=, short-plural=}
\DeclareAcronym{sds}	{short=SDS, 	long=spoken dialogue system, 						long-plural=s, short-plural=s}
\DeclareAcronym{smo}	{short=SMO, 	long=sequential minimization optimization, 			long-plural=, short-plural=}
\DeclareAcronym{svm}	{short=SVM, 	long=support vector machine, 						long-plural=s, short-plural=s}
\DeclareAcronym{tds}	{short=TDS, 	long=task-driven system,							long-plural=s, short-plural=s}
\DeclareAcronym{tts}	{short=TTS, 	long=text-to-speech,								long-plural=, short-plural=}
\DeclareAcronym{va}		{short=VA, 		long=voice assistant, 								long-plural=s, short-plural=s}
\DeclareAcronym{vacc}	{short=VACC, 	long=Voice Assistant Conversation Corpus, 			long-plural=, short-plural=}
\DeclareAcronym{vh}		{short=VH, 		long=virtual human, 								long-plural=s, short-plural=s}

\usepackage[	
	natbib=true,
	style=authoryear-comp,
	hyperref=true,
	backend=biber,
	maxbibnames=99,
	giveninits=true,
	uniquename=false, % init
	uniquelist=false,
	maxcitenames=2,
	mincitenames=1,
	parentracker=true,
	url=true,
	doi=true,
	isbn=true,
	eprint=true,
	backref=true,]
	{biblatex}
\addbibresource{dissertation.bib}

% add a background picture on the cover page
% --------------------------------------------
\newcommand \AlCentroPagina [1]{
    \AddToShipoutPicture *{\AtPageCenter {
            \makebox (0,0){\includegraphics
                [width=0.95\paperwidth]{#1}}}}}

% Cover Page
% ----------
\title{\vspace{-3.2cm} \texttt{Ph.D. Dissertation}\\ \rule{\linewidth}{0.5mm}\\
    \textsc{\Huge Integrating Phonetic Convergence Capabilities into Spoken Dialogue Systems
        \\\rule{\linewidth}{0.5mm}\\[0.5cm]}} \author{\LARGE
    \textbf{Eran Raveh}\\[1cm]
    \Large Universit\"at des Saarlandes\\
    \\[1cm] \Large \textit{Submitted in partial fulfillment}\\
    \Large \textit{of the requirements for the degree of}\\[0.5cm]
    \Large \textit{\textbf{Doctor of Philosophy in}}\\
    \Large \textit{\textbf{Computational Linguistics}}\\
    [1cm] \Large Supervisor: Dr.\ Ingmar Steiner\\[2cm]}
\date{\today}
%\titlepic{\includegraphics[scale=0.7]{Figures/logo_ekut.png}}
\AlCentroPagina{uds_logo_trans.png}
