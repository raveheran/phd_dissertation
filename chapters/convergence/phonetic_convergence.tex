\chapter{Phonetic Accommodation}
\label{chap:phonetic_convergence}

\lettrine{I}{n} this chapter, the concept of accommodation is introduced.
Types of linguistic changes and the related terminology used in this work are explained as well.
Finally, a survey of works on phonetic convergence in \acl{hci} is presented and discussed to lay the ground for the contributions of this work.

\pagebreak

\acresetall

\section{\Acl{cat}}
\label{sec:communication_accommodation_theory}

Communication is a fundamental part of life.
In addition to offering a means for exchanging information and expressing emotions and desires, it also exhibits salient markers of social membership.
Human communication is a complex concept concealing many facets and sub-processes within it.
This Complexity stems from two main aspects:
First, each individual is unique and communicates differently -- in general as well as across specific interactions -- based on inherent traits, emotional state, personal preferences, situational circumstances, and more.
Moreover, an interlocutor may belong to or represent a certain group (e.g., a social group or an organization), which may also influence the nature of the exchange.
Secondly, some forms of communication, in particular face-to-face, harness combinations of modalities.
This results in a large amount of information one needs to process in real-time to achieve efficient communication.
Furthermore, despite common social conventions, not every person perceives and processes this information the same way, which requires all interlocutors to be attentive as to how they comprehend others and vice versa.
Therefore, each exchange is unique and shaped by various personal and environmental factors, which often makes interactions hard to analyze and predict (ass discussed in \cref{subsec:dialogue_is_hard}).
To cope with such highly variable dynamics, people must have some way to know -- whether unconsciously, intuitively, or deliberately -- how to adjust their communicative behavior with respect to the other interlocutors in the conversation.

\Acf{cat} is a theoretical framework of communication which aims to explain the personal and social motivations in verbal and non-verbal human communication.
A core motivation in \ac{cat} is \emph{social distance} \citep{Giles1973mobility, Giles1991CAT, Giles2007CAT}, which suggests that to reduce social distance, a speaker may converge toward the conversation partner(s), whereas divergence would lead to an increase in social distance.
When and how social distance should be altered depends on the social class of the speakers, their formal role and personal goals in the interaction, etc.
Particularly, these changes occur with respect to how the other conversation partners speak, from overall psychological, social, and linguistic behaviors to specific features (like those introduced in \cref{sec:linguistic_accommodation}).
For example, different communication styles would probably be utilized when speaking to a childhood friend, a colleague, or a company's executive.
In each of these situations, the speakers would likely use different language registers (e.g., slang words and politeness markers) in an attempt to fit into the social groups and become closer to its members.
Another example is the use of \enquote{elder language} when talking to old people (e.g., using slower speech, extended use of hand gestures, etc.), to make it easier for them to understand the speaker.
Such adjustments can be, to some extent, conscious, but often occur automatically.
Ideally, both speakers eventually find their comfort zone during the conversation.
However, there exists the notion of over-accommodation, which is usually caused by very conscious speakers that intentionally exploit this phenomenon.
If realized by a conversation partner, this might be perceived as pretentious or fake behavior and even mockery.
Similarly, accommodation can serve a speaker with audience design when talking to a larger group of people.
The adjustments may be reflected simultaneously in different modalities, like hand gestures, body posture, facial expression, eye gaze, lexical choices, and speech.
This work concentrates on the latter.
While \ac{cat} advocates these changes, it is worth mentioning that there are other further frameworks that explain them as well.
For instance, the Interactive Alignment Model \citep{Pickering2004behavioral, Pickering2013integrated} offers a model where adjustments in communicative behavior are described as \emph{alignment} as opposed to \emph{accommodation}, hinting that the process is rather one-sided and uni-directional (see \cref{tab:variation_types}).
Despite this difference and others, all these frameworks describe a process where the similarity between interlocutors increases or decreases with respect to certain features over the course of their communication.
\Cref{subsec:variation_types} sheds more light on the difference between the numerous terms used to describe this process and how they are used in the literature.

\subsection{Variation types -- terminology}
\label{subsec:variation_types}

The term \enquote{convergence} is a central notion in this work.
This term's definition in this work is different from its meaning in other fields, like \ac{ml}.
Moreover, various terms are used in the literature synonymously to describe the same phenomenon, although there are often subtle differences or emphases on certain facets of it.
Disambiguating these terms should help to avoid confusion and allow a more fine-grained description of such processes.
The core meaning of convergence is explained here in detail, followed by a list of related terms and explanations about their use in the literature.

The verb \textit{to converge} is defined in Cambridge Dictionary\footnote{\url{http://dictionary.cambridge.org/dictionary/english/converge}} as \enquote{moving toward, or merging into, the same point (e.g., roads)}.
A more non-materialistic definition is provided as well: \enquote{If ideas and opinions converge, they gradually become similar}.
Another, somewhat more general, definition to the noun \textit{convergence} is given in Merriam-Webster dictionary\footnote{\url{https://www.merriam-webster.com/dictionary/convergence}}: \enquote{the act of converging and especially moving toward union or uniformity}.
As these definitions suggest, the core idea of this concept is two (or more) entities that change (potentially in different degrees) toward some physical or abstract point and ultimately meet.
In some time-dependent cases, like spoken interactions, there might not be enough time for the speakers to meet, but they can still become more similar to one another nonetheless.
This matches the definition of \emph{phonetic convergence} proposed by \citet{Pardo2006phonetic}: \enquote{[\ldots] increase in segmental and suprasegmental similarities between two speakers}.
This also resembles the definition suggested by \citet{Xia2014prosodic}: \enquote{[\ldots] behaviors become more similar over time}, with \enquote{behaviors} referring to different modalities of communications in conversation, e.g., facial expressions, gestures, lexical choices, etc.
Other terms are sometimes used to describe similar processes or different aspects, and it is important to make the distinction between them.
Note that some of the terms are used interchangeably, as synonyms, or as hypernyms and hyponyms in other works.

The list presented here aims to unequivocally distinguish these terms and grant them useful relations to offer a meaningful common terminology, at the very least within the scope of this work and hopefully in the community of this research area in general.
\cref{tab:variation_types} summarizes the terms comparison.

\begin{table}[t]
	\centering
	\caption[Comparison of variation types]
		{Comparison of variation types properties.
		\todo[inline]{explain each column}
		A \partick\ sign indicates that the property is fulfilled to some extent, or that it may be implied or assumed in some cases.
		The \enquote*{*} signs mark the terms most widely used in the literature.}
	\label{tab:variation_types}
	\begin{tabularx}{\linewidth}{Xcccc}
		\toprule
								& mutuality & directionality & intention/awareness & defined target\\
		\midrule
		accommodation*			&	\partick	&				&				&				\\
		\rowcolor{lightgray}
		convergence*			&	\partick	&	\tick		&	\partick	&				\\
		mimicry					&	\partick	&	\tick		&	\partick	&	\partick	\\
		\rowcolor{lightgray}
		proximity				&	\tick		&				&				&				\\		
		synchrony				&	\tick		&	\partick	&				&				\\
		\rowcolor{lightgray}
		mirroring				&	\tick		&	\tick		&				&	\tick		\\
		coordination			&	\tick		&	\partick	&	\tick		&	\partick	\\
		\rowcolor{lightgray}
		assimilation			&				&	\tick		&	\partick	&	\partick	\\
		adaptation*				&				&	\partick	&	\tick		&	\tick		\\
		\rowcolor{lightgray}
		chameleon eff.			&				&	\tick		&				&	\partick	\\
		priming					&				&	\tick		&				&	\tick		\\
		\rowcolor{lightgray}
		entrainment*			&				&	\tick		&	\partick	&	\partick	\\
		alignment				&				&	\tick		&	\partick	&	\tick		\\
		\rowcolor{lightgray}
		imitation				&				&	\tick		&	\tick		&	\tick		\\
%		int.sp.\ influence		&						&	\tick		&	\partick				&	\partick			\\
		\bottomrule
	\end{tabularx}
\end{table}

\begin{description}
	\item[accommodation] -- is often used in the literature as an overarching term derived from its meaning in \ac{cat} and includes any dynamic mutual changes during an interaction.
	Concretely, it includes both convergence and divergence, but also other forms of change caused by influences of other interlocutors.
	An important aspect of any accommodative effect is that it is a result of external input.
%	Therefore, \acp{sds} with accommodative capabilities are described as \emph{responsive} in \cref{subsec:accommodation_levels} (and see example in~\cref{fig:validation_sensitivity}).
	
	\item[convergence] -- As explained above, convergence means an increase in similarity between speakers \citep[see Figure 1 in][]{Levitan2011measuring}.
	Each speaker can account for a different \enquote{amount} of the overall convergence effect, based on the speaker's tendency to converge in the specific interaction and in general.
	There may be great individual differences in this tendency, as shown in the experiments in \cref{part:experiments}.
	This tendency is referred to as the speaker's \emph{sensitivity} to external changes in \cref{sec:parameters}.
	Convergence can result in the speakers' production values meet somewhere in the middle (as in \cref{fig:accommodation_types}) or closer to one of them, depending on the social dynamics in the interaction.
	Similarly, \textbf{divergence} is the reversed effect, i.e., when a decrease in the similarity between the speakers occurs.
	Divergence is generally less common in \ac{hhi}, but may occur in competitive (as opposed to collaborative) scenarios, or when a speaker, intentionally or not, thrives to deliberately be distinguished from the other.
	
	\item[entrainment] -- is presented by \citet{Brennan1996lexical} as the possibly aware influence of an interlocutor on another, e.g., imposing lexical units in an interaction, with emphasis on the one-sided nature of the process.
	This means that one of the interlocutors established the use of an aforementioned term which created a bias by the conversation partner to employ it as well.
	The change is sometimes seen as categorical (similar to \textit{priming}), which is a key difference from the definition of convergence here.
	This is especially relevant in \ac{hci}, as it is much easier for a computer, due to the vocabulary problem \citep{Furnas1987vocabulary}, to establish such uses.
	\citet{Lopes2013automated} describe entrainment as imposed by a human speaker, where a computer-based interlocutor follows the lexical choices of the user.
	This term is sometimes also used as a static measure of changes in an interaction (cf.\ \cref{subsec:limitations_of_did}), as in \citet{Levitan2013entrainment}.
	Conversely, \emph{convergence} refers here to the potentially mutual dynamic measure, with the difference being comparing discrete timestamps in the interaction \citep[for example, between two halves of a session, as done by][]{Xia2014prosodic} versus changes occurring gradually over its entire length.
	In addition to all the above, entrainment is often synonymous with convergence or accommodation in more technical fields.
	
	\item[synchrony] -- refers to an ongoing process where the interlocutors are changing their behavior similarly, i.e., synchronously.
	Importantly, this applies to the relative changes in each speaker, but not to any absolute values.
	That is, the distance is generally maintained while the individual ranges may differ \citep[see example in \cref{fig:synchrony_switchboard} on page~\pageref{fig:synchrony_switchboard} and Figure 1 in][]{Levitan2011measuring}.
	When one of the speakers is leading the synchrony, it becomes \emph{lagged}, as shown in \cref{fig:accommodation_types}.
	Lagged phenomena and the speaker leading them are often determined using some correlation measure, like Peasrson's coefficient in \citet{Edlund2009pause, Xia2014prosodic}.
	A deeper analysis of such an accommodation effect is presented in \cref{sec:analysis_hhi}.
	
	\item[adaptation] -- refers to the process of making intentional changes that suit certain conditions or situations.
	As such, this term stresses the ambition of these aware modifications in vocal behavior made by an interlocutor to be more similar to a known and defined target exhibited by another speaker.
	\citet{Kang2010emergence, Hwang2015phonetic} both examine the phonetic adaptation process that occurs gradually when encountering a new sound environment, to which speakers want or are expected to adapt.
	In these cases, the changes are also likely to be maintained outside the scope of a specific interaction.
	Adaptation is also used to describe the technical capability -- or more often the lack thereof -- in a machine to match its speech production to a user interacting with the system.
	This fits the definition of adaptation being intentional and having a well-defined target to adapt towards.
	However, this is still a very limited feature due to the current state of \ac{tts} technologies.
	The integration of such capabilities into \acp{sds} and the involved challenges are discussed in \cref{sec:adaptive_spoken_dialogue_systems,chap:convergence_module_for_sdss}.
	
	\item[priming] -- is similar to entrainment, but usually works in a larger scope, in terms of both time and the degree of change.
	It is typically used in the context of psycholinguistics.
	As opposed to entrainment, e.g., on the lexical level, priming can influence not just the use of one specific term, but a whole semantic field.
	For example, it was shown in experiments by \citet{Meyer1971facilitation, Schvaneveldt1973retrieval} that people respond faster to words from a specific semantic field after being exposed to other words from it over a long timespan.
	% (i.e., different words that are semantically related).
	In more general terms, priming changes the likelihood of a person to use specific behavior, typically on some semantic or syntactic level.
	The temporal scope is different as well.
	Priming can have an effect in a longer-term than a single interaction, ranging from multiple interactions across several hours, days, and up to years.
	Therefore, this term is often used when talking about language change in children \citep[see, e.g., ][]{Huttenlocher2004syntactic, Wansink2012would}.
	Like entrainment, priming also usually describes a categorical change \citep[cf.][]{Reitter2006computational, Pace2013concept}.
	
	\item[assimilation] -- implies a one-sided process, where one side changes in a certain way to match the other.
	It is typically used to describe an accommodative process with the motivation to associate oneself with a social group by adopting its vocal characteristics.
	Therefore, assimilation is seen as a change that occurs as a result of a specific situation or context, e.g., public speeches, as shown by \citet{Ohala1990phonetics} and discussed in \cref{subsec:sound_change}.
	This resembles the use of this term in phonology to describe a sound that changes to become similar to another sound with respect to a certain property \citep[see examples in][pp.~89-98]{Hall2011phonologie}.
	%	e.g., when voiceness of a sound changes to match this of the sound adjacent to it.
	%	Note that assimilation is only one of the phonological processes (others would be dissimilation, epenthesis, elision, and more), but since it's the once describing two things becoming more similar to each other, it is the one that might be used in this context.
	
	\item[alignment] -- is a term derived from the Interactive Alignment Model \citep{Pickering2004behavioral} and describes an increase in similarity between speakers.
	It describes a process similar to assimilation, but with a certain motivation behind it.
	While in assimilation the change is measured with respect to a social group or speech style, alignment is often used to describe the adoption of speech characteristics of a speaker in a specific conversation.
	Specifically, it is claimed to contribute to the overall ease and success of conversations by the interlocutors to behave similarly on different linguistic levels \citep{Garrod2009joint}.
	
	\item[mirroring] -- is a cognitive tool that aims to produce an output that follows some input.
	A.k.a.\ \enquote{mirrored equivalence} \citep{Messum2015creating, Messum2007mirroring}, this process differs from imitation by the lack of aim toward a well-defined target, but rather an internal representation of the similarity between the speakers.
	This may lead to a less planned and sometimes automatic effect of becoming closer to an interlocutor, where the specifications of the change do not necessarily match the input target.
	Less commonly, this term also refers to an effect opposite to synchrony, where the directions of the respective changes are reversed instead of similar (as if a mirror was put between them).
	Additionally, mirroring may also refer to an effect similar to mimicry, but with greater emphasis on speech learning and acquisition \citep[e.g.,][]{Yoshikawa2003constructivist}.
	
	\item[mimicry] -- is the tendency to behave, or speak, broadly like someone else.
	The emphasis here is on the \emph{general} inclination, which hints it can be done unintentionally, or, alternatively, be deliberate, but without the goal to perfectly match the target (as opposed to imitation).
	% inspired by https://www.voices.com/blog/mimicry_vs_imitation/
	\citet{Gueguen2009mimicry} demonstrate how mimicry can earn the mimicker more favorable judgment in social interactions.
	This is explained by mimicking creating a greater feeling of affiliation and rapport in communication, or with the more positive evaluation of the mimicked person due to an enhanced familiarity established by the mimicker.
	\citet{Parrill2006seeing} show just how natural mimicry in \ac{hhi} is using an experiment in which participants' behavior was affected merely by watching mimicking takes place in another conversation.
%	reference, mention \textit{Chameleon effect}\ldots
	
	\item[imitation] -- is similar to mimicry, but is done intentionally and with the goal to match as closely as possible to a target.
	In speech, it emphasizes the speaker's intent to completely match the auditory input \citep[cf.][]{Gueguen2009mimicry}.
	% inspired by https://www.voices.com/blog/mimicry_vs_imitation/
	The attempt to \emph{deliberately} repeat and \emph{accurately} replicate another speaker's productions distinguishes imitation from mirroring and mimicry.
	
	\item[chameleon effect] -- is a more general account of mimicry, but with complete non-conscious adoption of an interlocutor's behavior \citep{Chartrand1999chameleon}.
	It is a term from social psychology typically associated with a comprehensive change in multiple modalities and in the overall mannerisms.
	\citet{Gueguen2009mimicry} focus on the social aspects of this effect and how it can change social judgment and attitude toward the speaker.
	Both works describe it as "monkey see, monkey do", which emphasizes the automaticity of it.
	Moreover, they state that it is a learning mechanism for children and for humans in general before the development of unified languages \citep[][p.~256]{Gueguen2009mimicry}.
	
	\item[proximity] -- describes general closeness between interlocutors \citep[as illustrated in Figure 1 in][]{Levitan2011measuring}.
	This does not imply specific distance thresholds or absolute value ranges.
	Furthermore, this is a rather passive, potentially merely circumstantial, state, which does not involve a defined vocal target or an aspiration to match it.
	The proximity at a specific point in time (e.g., around the start of a conversation) can be used as a reference point for other measures.
	
	\item[coordination] -- implies cooperation -- either seeming or real -- between interlocutors.
	Increase or decrease of communication features is in this case a side of effect of this coordination.
	This provides another point of view on the process, namely not to examine the speakers' collaboration based on common changes, but looking at these changes as part of this collaboration.
	
%	\item[inter-speaker influence] -- a general term to describe a change by a human interlocutor caused by another.
\end{description}

\afterpage{%
	\begin{landscape}
		\begin{figure}[t]
			\centering
			\vspace*{-2cm}
			\includegraphics[width=\linewidth]{synchrony_switchboard_2005}
			\caption[Example of pitch synchrony between two speakers]
				{Example of pitch synchrony between two speakers in conversation number \texttt{2005} of the SwitchBoard corpus \citep{Holliman1992Switchboard}.
				The turns of speaker A (red; top) and speaker B (blue; bottom) are marked by the respective vertical bars.
				The circles in the corresponding colors show the individual production values.
				The \acf{loess} trend lines show the overall synchrony (with slight convergence) between the speakers over the course of the conversation.
				The speech balance measure at the bottom is a value between 0 and 1 indicating how balanced is the overall speaking time between the two speakers (the higher the more balanced, and the interactivity measure indicates the frequency of floor change without (and with) backchanneling (0 points to only long monologues and 1 to floor change after every turn).
				These measures are further explained in \cref{subsec:results_hhi} on page \pageref{eq:speech_balance}.}
			\label{fig:synchrony_switchboard}
		\end{figure}
	\end{landscape}
}

\section{Linguistic accommodation}
\label{sec:linguistic_accommodation}

Accommodation occurs on various linguistic levels in \ac{hhi}.
Relative salient changes may occur on the lexical level when one interlocutor shifts his lexical choices to match those of another.
This is more likely to happen when a lexical entity has multiple commonly used alternatives, like synonyms or different names.
For example, \citet{Jucks2008lexical} show how healthcare experts try to match their wording to patients in written inquiries.
In another experiment, \citet{Friedberg2012lexical} found increasing lexical similarity over the course of spoken discussions among students groups with better performances.
\citet{Racz2020morphological} even extended the analysis to morphological forms, suggesting that morphological convergence occurs and creates generalizations in memory in real-time.
Other examples of lexical convergence have also been found in \ac{hci} when looking for information \citep{Lopes2013lexical} or when playing \citep[][and see \cref{subsec:previous_work}]{Bergqvist2020nontrivial}.
In all these experiments, it was shown that lexical convergence had a positive effect on task performance.

The focus in this work is on accommodation occurring in speech-related features, i.e., \emph{vocal accommodation}.
A core difference between lexical and vocal accommodation is the mechanism for defining two instances as the same.
For example, the word \enquote{window} is written the same way regardless of who writes it, which makes it easy to associate tokens with the same type.
However, vocal features are strongly dependent on the speaker.
The signal representing a word would be different depending on the speaker's physiological properties (like vocal tract size), speaking rate, voice intensity, intonation, and many more.
Moreover, it is very unlikely that even the same speaker would pronounce the same word in the same manner.
This is especially true in conversation taking place in different settings and environments, but also for two successive utterances in the same conversation.
Additionally, the relative differences, e.g., in vowel quality or speaking rate, also differ from one person to another, which entails that more evidence might be needed for a perceivable target for accommodation to emerge.
An exception to these differences could be categorical phonetic differences, where each category may be perceived as a separate entity (similarly to the lexical case), making it easier to define the target (and see \cref{subsubsec:target_features_HCIConv,subsec:speech_manipulation}).
This great variety in spoken language makes the accommodation process more complex, as the targets might not always be well-defined and static.
%(comparing with a lexical entity that would always be recognized as the same).

Vocal accommodation has been found in both segmental \citep{Pardo2010conversational, Smith2007prosodic} and suprasegmental \citep{Walker2015repeat, Shockley2004imitation} phonetic features and
both in conversational \citep{Pardo2006phonetic, Lewandowski2012talent, Weise2018looking} and non-conversational \citep{Babel2014novelty, Shockley2004imitation} scenarios.
There is evidence for it being both an internal mechanism \citep{Pickering2004behavioral} and socially motivated \citep{Kim2011phonetic, Giles1991CAT, Babel2010dialect}.
For instance, phonetic convergence \citep{Giles1973mobility} or divergence \citep{Bourhis1977distinctiveness} is triggered by decreasing or increasing social distance between interlocutors, respectively.
\Citet{Baumann2020how} even shows that feedback given by conversation partners can influence the convergence process.
\citet{Aubanel2020speaking} found that speakers accurately and consistently converge to each other's \ac{f0} in scripted read-aloud dyadic conversation on a turn-by-turn basis.
This shows the ability to track changing \ac{f0} both in perception and production.
Similar effects were demonstrated by \citet{Pardo2013measuring} with respect to multiple phonetic features.
The study conducted by \citet{Babel2012role} supports the importance of \ac{f0} in convergence effects by showing that filtering out the \ac{f0} frequencies in the signals the participants hear leads to reduced convergence.
\citet{Schweitzer2017visibility} show that convergence effects are stronger when conversation partners can speak but not see each other, and divergence occurs more when they can also see each other while talking.
Moreover, the effects were stronger depending on the degree of likability between them.
This shows that accommodation can be influenced by other, non-linguistic factors of the conversation.
According to \citet{Lehnert2020relationship}, language skill level and greater \ac{f0} expressiveness also influence the degree of convergence.
Accommodation effects were also found in intensity levels of speakers:
In an experiment where participants heard an interviewer in different levels of vocal intensity, \citet{Natale1975convergence} shows that their intensity was generally changed accordingly.
In a second experiment, the degree of intensity convergence could be predicted by a social desirability test, which stands in line with the finding of \ac{f0} accommodation.
A third commonly studied feature is \acl{ar} (\acs{ar}; or the related measure \emph{speaking rate}).
In an exemplar-theoretic view in mind, \citet{Schweitzer2016exemplar} investigated how syllable frequency influences the degree of change.
Evidently, stronger effects were found in more frequent syllables, which supports this view.
Relatedly, \citet{Edlund2009pause, Xiao2015analyzing, Cohen2017converging} introduce ways to measure temporal prosodic changes, like pauses, in conversation, which affect the speaking rate.
\citet{Levitan2011measuring, Local2007phonetic} found convergence effects in all these three features (and others) in collaborative games and everyday scenarios.
Such a wide variety of evidence suggests that, even in the vocal modality alone, accommodation is reflected in various ways in \acp{hhi}.
These features are investigated in this work in both \ac{hhi} (\cref{chap:conv_analysis}) and \ac{hci} (\cref{chap:speech_variations_in_hhci}) contexts.
In addition to these, other factors and phonetic features were investigated for accommodation, such as voice quality \citep{Borrie2017conversational}, voice onset time \citep{Nielsen2011specificity}, and other timing-related phenomena \citep{Putman1984conception}, second language proficiency \citep{Law2020convergence}, interlocutors' sex \citep{Levitan2012acoustic, Bailly2014assessing}, perceived attractiveness \citep{Michalsky2017pitch}, word frequency \citep{Nenkova2008high}, and more.
A survey of methods for measuring accommodation in \ac{hhi} can be found in \citet{Lewandowski2019phonetic, DeLooze2014investigating}.

\subsection{Long-term and short-term sound changes}
\label{subsec:sound_change}

All the sound changes discussed in this work are short-term changes occurring over the timespan of a single, even if long, interaction (\cref{chap:conv_analysis}) or multiple sequential interactions (\cref{chap:speech_variations_in_hhci}).
Similarly, the models and applications in \crefrange{chap:computational_model}{chap:web-based_responsive_spoken_dialogue_system} are also designed to handle accommodation-related changes within the scope of individual interactions.
However, sound changes may occur continuously over long periods of time -- even years or decades.
While short-term accommodative changes can be ascribed to local social influences of specific interlocutors (as discussed in \cref{sec:communication_accommodation_theory}), long-term changes may stem from larger cultural influences and independent evolution of a person's speech.
The changes' source can be both the speaker and the listener \citep[][pp.~176-187]{Ohala1989sound} and is typically caused by confusion or correction \citep{Ohala1993phonetics}.
The latter is more relevant, for instance, for short-term assimilation, as demonstrated in \citet{Ohala1990phonetics}.
The former, however, is more dominant in long-term changes and cross-language influences, e.g., when sounds in one language are replaced by similar sounds in another in loanwords.
This is also related to mutual influences of speakers in the evolution of a language, or the way different speaking styles can be created within a language to mark cultural, regional, and social differences.
These reasons and others are explained from the phonetic point of view in \citet{Sweet1874history}.

Such long-term changes can also occur in the speech of a single person.
\citet{Harrington2007evidence} examined vowel changes in the Queen's pronunciation in her annual Christmas messages from 1952 to 2002.
Some gradual changes were, indeed, found, but could not be ascribed to age or varying speech style.
The author, therefore, considered it an independent, long-term sound change of an individual.
Contrarily, in the pronunciations around the 1980s assimilation was found towards accents associated with younger people of lower classes \citep{Harrington2000does, Harrington2000monophthongal}.
This suggests a potentially aware audience design from the Queen's side \citep{Bell1984language}.
Such use of communication falls under the social motivation of \ac{cat}, although the interactions were one-sided.
In this example, the changes were not a result of interactions over a long period of time.
However, this may also happen between people who regularly speak with each other for many years.
This is relevant for \acp{hci} that are designed to last a very long time, like \acp{pa} or social companion (see \cref{sec:types_of_sdss}).
Therefore, in such systems, accommodation effects should be taken into account as well, but the modeling approach could take advantage of the fact that there is a much longer timespan for the changes to shape.
This can be used, for example, for accumulating more evidence before deciding on the appropriate change from the system side (cf.\ exemplar approach in \cref{chap:computational_model} and floor-change approach in \cref{chap:statistical_model}).

\subsection{Measuring accommodation} % Limitations of \acl{did} measures
\label{subsec:limitations_of_did}

Conversations are complex processes that require some expertise and quantitative analysis tools for detecting and isolating specific patterns in them, especially since effects may have long, non-linear relations.
A behavior is a complex collection of conducts of a person, particularly those towards a certain environment.
The realization of one's behavior over the course of a conversation can communicate information regarding one's state and goals, especially with this deviates from the individual's expected behavior.
The behavior leads to reactions to environmental input and can be modified due to reinforcements from the environment or self-directed motives.
These modifications can occur over time quickly or slowly, consciously or unconsciously, and to a greater or lesser extends, which makes them dynamic and thus cannot be defined discretely.
All the above is true for vocal behaviors as well, which are expressed in spoken interactions.
Specifically, vocal accommodation reflects dynamic changes during a conversation that can be affected by the external speech input of other interlocutors.
It can therefore be beneficial to \textbf{move away from value comparisons in favor of behavior descriptors} to depict an interaction.
This is done by analyzing spoken interactions as entire events with a continuous temporal dimension as opposed to a comparison between discrete points in time.
Examining the whole conversation can help, for example, to determine who was leading the changes or when more accommodation occurred.
In this work, such analyses were utilized in \cref{chap:conv_analysis} to determine the leading speaker in each conversation and in \cref{chap:speech_variations_in_hhci} to expose the ongoing changes in the human speaker's productions.

The way speakers accommodate to each other is very unlikely to be linear from beginning to end and can change throughout the conversation.
Therefore, comparing the differences between two discrete, distant points in time might miss or oversimplify some dynamics that occurred between them.
Moreover, comparisons like that usually take the view of one speaker instead of looking at the conversation as a complete entity.
Quantitative analyses often leave accommodation hidden or overly smoothed if done on a turn-by-turn basis or by splitting it arbitrarily into two or more parts (often equally-long, non-overlapping time intervals) and directly comparing them using raw values as in \citet{Heldner2010pitch, Rahimi2018weighting, Ibrahim2019fundamental}.
\citet[][p.~15]{DeLooze2014investigating} point out that in these cases it is assumed that accommodation is a strictly local phenomenon where a speaker's utterance is linked exclusively to the other interlocutor's immediate preceding utterance.
Such analyses result in a linear, static representation of the conversation's evolution, from which generalized conclusions are hard to draw.
They might even be inaccurate or misleading, especially when both speakers change their output over time, as demonstrated by \citet{CohenPriva2019limitations}.
That work refers specifically to \ac{did} measurements, where pairs of two discrete points in time are compared to measure accommodation between speakers using the following distance formula\footnote{Note that this way of measuring \ac{did} is commutative and therefore doesn't measure the changes from the view of a specific speaker \citep[unlike,~e.g.,][p.~3]{CohenPriva2019limitations}.}
%
\begin{equation}
	\label{eq:did}
	DiD_{\vec{i},\vec{s}} \coloneqq \sqrt{(\vec{i}_{t + 1}^n - \vec{s}_{t + 1}^n)^2 + (\vec{i}_t^n - \vec{s}_t^n)^2} \equiv \lVert \vec{i} - \vec{s} \rVert_n.
\end{equation}
\eqname{\Acl{did} measure (simplified)}
\noindent
%
\cref{fig:accommodation_types} shows the interaction between two speakers' productions in a hypothetical conversation.
By merely looking at the plot, it is clear that the two speakers do not sustain the same behavior throughout the conversation.
For example, between marks A and B the orange speaker's values go remarkably downwards (though not linearly), while the green speaker remains roughly stable -- i.e., \emph{divergence} (although this can also be seen as an independent change).
Subsequently, the distance is generally maintained between B and C.
Between C and D \emph{convergence} occurs \textbf{in both speakers}.
Lastly, lagged \emph{synchrony} can be observed between marks D and E.
If compared directly, the lagging might make the value changes look somewhat random.
However, looking at the entire segment, it's clear that the change is similar and is led by the green speaker.
None of these mutual behaviors can be captured by a \ac{did}-based approach.
For instance, comparing the beginning (mark A) and end (mark E) of the conversation would lead to the conclusion that there was no change in the values (illustrated by the corresponding dashed lines).
Similarly, splitting the conversation into two halves (A to D and D to E) would make it look like the changes were symmetrical, missing the obviously different behaviors of the speakers in these two halves.
Additionally, the directionality of the convergence between marks C and D will be missed by simple \ac{did} distance measures, which might give the impression that the changed rooted from only one of the speakers.
This could be satisfactory when it can be assumed that such changes can only occur in one speaker, like when talking with some computer-based interlocutor like a \ac{pa} (see \cref{sec:convergence_to_natural_and_synthetic_stimuli,sec:vacc}), but not when both speakers' productions may be flexible, like in \ac{hhi} (\cref{sec:analysis_hhi}) or ultimately adaptive \acl{sds} (\cref{sec:adaptive_spoken_dialogue_systems}).
This limitation can be compensated to some extent by examining the \emph{relative} occurring changes \emph{gradually} (\cref{fig:condition_convergence_comparison,subsec:temporal_analysis}).

%Methods used in time series analysis are utilized in this paper to detect and model long-term relations within a conversation.
%This approach offers different types of analyses and emphasizes the temporal aspect of spoken interactions.
%The first is \ac{crqa}, which is proportionate to the conversation's length.
%It measures the \emph{mutual} similarity changes of the speakers throughout the entire conversation in term of recurrence.
%Additionally, it tells when the recurrence segments sustained for longer periods and which speaker led them.
%\cref{sec:capturing_patterns_in_time_series} explains how this method is used to differentiate the dataset introduced in \cref{sec:dataset_and_feature_extraction} to define different profiles.
%Another method is neural autoencoder (\cref{subsec:dim_reduction}), which can learn non-linear representations in a latent space.
%This technique is good for creating generalized models based on many conversations and speakers.
%By using \ac{vae}, these models become generative and can therefore be used for producing varying speech output of a system fitting to its chosen profile or combination of profiles.
%Finally, \acp{gp} add stochastic Gaussian variability to the yielded output based on the function distribution derived from the models.
%Variations of the base behavior are produced that way.
%This is demonstrated in \cref{subsec:generation_gp_kriging}.

%As discussed in \cref{subsec:conversation-level_accommodation_behavior}, \ac{did} and sequential methods for measuring vocal changes (e.g., accommodation) rely on the chronological, turn-by-turn order of the interaction and their scope is limited to detect local phenomena only.
%Non-linear methods, contrarily, are not dependent on the chronological order of the interaction's turn and can therefore find long-term, more distant relations between the speakers with respect to the target feature.
%For instance, that an accommodation process (or generally closer values) occurred at some point in the beginning and continued at a later time, or that there was a periodic pattern of convergence and divergence throughout the interaction.
%Such phenomena point to more general, conversation-level properties that do not rely on the unfolding chronological order of the turns.
%This is especially useful for long interaction, where various patterns occur a more general view may be insightful (see example in \cref{fig:accommodation_types}).
%
%We take here a different view on the structure of the conversation, namely referring to it as a set of \emph{time series}.
%Time series is a sequence of chronologically ordered values sampled from some data stream along equal time intervals.
%Due to the way the dataset presented in \cref{sec:dataset_and_feature_extraction} is analyzed, the extracted features can be treated as couples of \emph{time series}, one for each speaker in each conversation.
%\todo{examples of where speech is treated as TS (ASR?)}
%Because of the nature of these features (and perhaps any spontaneous, interactive linguistic feature), the resulted time series can be assumed to be non-seasonal and non-stationary.
%\todo{might want to remove this sentence to not commit to this type of data}
%This view on the data enables a new assortment of analysis methods, from which \acf{crqa} and \acf{gp} are demonstrated here (\cref{subsec:crqa,subsec:generation_gp_kriging}).
%
%This recognition as time series opens new methodological possibilities for examining the evolution of these features and their realizations by different speakers in an interaction over time.
%Such methods include, among others, autocorrelation for examining serial dependency and forecasting for transferring information about the time series across time.
%Non-linear analysis methods for time series include, for example, noise reduction and non-linear prediction.
%We utilize here another method that uses phase-space embedding, which describes temporal evolution of trajectories of a dynamic system by projecting their embedding onto some common space.

\begin{figure}[t]
	\centering
	\includegraphics[width=\linewidth]{accommodation_types}
	\caption[Different accommodation types in a conversation]
		{The evolution of a feature's values produced by two speakers represented by the green and orange solid lines.
		The dashed lines connect the corresponding initial and final values of each speaker.
		The letters A to E mark timestamps with behavioral changes.
		Each caption describes the behavior between the two corresponding marks.}
	\label{fig:accommodation_types}
\end{figure}

\section{Vocal accommodation in human-computer interaction}
\label{sec:phonetic_convergence_in_hci}

%\subsection{What is an interaction?}
%\label{subsec:what_is_an_interaction}

\subsection{Verbal interaction with computers}
\label{subsec:verbal_interaction}

A \acf{hhi} is a mutual or reciprocal relationship between two (or more) interlocutors within a limited timespan.
This is also true for interaction with machines, though the beneficial side is typically the human speaker(s) while the machine is used as a tool to achieve the humans' goal.
In more modern applications, and especially when \ac{ai} is involved, the computer might also be programmed to \enquote{benefit} from the interaction as well, e.g., by acquiring information for future interactions or being able to finish a task more efficiently.
One type of interaction is a \emph{conversation}, where the communication is language-based.
This difference is more prominent in \ac{hci}, since there are ways to interact with machines without using written or spoken words, like using touchscreens, a computer mouse, or hand gestures.
This can be compared to non-verbal -- neither written nor spoken -- human communication, but purely non-verbal communication occurs more often with machines than with people.
The main reason for that is probably since machines are not, yet, capable of using language as freely and verbosely as humans.
The terms \enquote{interaction} and \enquote{conversation} are often used interchangeably in this work, since the interactions in question are spoken conversations.
Nevertheless, interactions refer to a more general concept of communication that might involve other components than speech while conversations focus on the language components of the communication.
% spoken interaction -- Although the linguistic content might remain the same, adds complexity due to variety or speech and the technological challenge of generating inteligble speech output.

Interestingly, humans almost always need to compromise on the way they interact with computers or learn new interfaces (like those mentioned above) even for performing simple tasks.
Even in the case of spoken communication, which develops early on in humans, compromises need to be made as to how to speak to the computer so that it understands the user's intention.
For any of the system types mentioned in \cref{sec:types_of_sdss}, users need to learn how to modify their speech so that they can properly use the systems (be it the speech style or wording), instead of the system being able to adapt to the user.
With the advances in speech technologies, this gap is shrinking, but there is still a way to go before computers will be able to understand and produce spoken language well enough for people to speak to computers the same as they speak to other human beings.
The topics in this work capitalize on this evolution, to see whether \ac{hhi} phenomena like vocal accommodation are transferred to \ac{hci} as talking to computers becomes easier and more common.
More generally, the question arises whether \ac{cat} holds for \ac{hci} as well.
This is supported, for example, by the \ac{casa} paradigm \citep{Nass1994computers, Nass2000machines},
which argues that humans apply similar social behaviors when interacting with computers because they ascribe human characteristics to them.
As a starting point, domain-specific systems with alternate turn-taking are easier for computers, as they take away a lot of the complexity of spoken language and reduce it to individual utterances that can be mapped to actions the systems support.
For example, a \ac{sds} for ordering train tickets will probably follow a very specific protocol and react only to a specific input, as opposed, for example, for a general-purpose system that could talk with the user about a planned trip and help booking tickets as part of a longer, general-purpose conversation (see \cref{tab:sds_types}).

\subsection{Previous work}
\label{subsec:previous_work}

\Cref{sec:linguistic_accommodation} discusses vocal accommodation in \ac{hhi}, but this phenomenon has been studied in \ac{hci} as well.
The key difference between the two settings is the lack of inert changes in computers.
Since accommodation is often ascribed to mutual social aspects (as explained by \ac{cat} in \cref{sec:communication_accommodation_theory}), this introduces a limitation on the computer's side.
Two main approaches are used to overcome this limitation in experiments:
Simulating a computer's output in a wizard-of-Oz setting and integrating basic accommodation capabilities into a \ac{sds}.
Wizard-of-Oz experiments have the advantage that the output of the computer-based interlocutor can be directly controlled by the experimenter, usually using pre-defined utterances.
This grants precision and control over the experiment, which makes it a very suitable approach for research.
However, preparing the experimenter's control interface and the utterances might be time-consuming, e.g., if they need to be recorded or manually manipulated in a certain way.
Another disadvantage is the disability to deviate from a pre-defined script covered by the prepared utterances, limiting the variety of interactions the simulated system can support.
\Acp{sds} that support at least some level of accommodation or real-time manipulation save the time and effort of creating stimuli prior to the experiment.
Though the quality of the output and the time required to generate it might be affected, this setting better represents real-world \acp{hci} and can be more flexible in different scenarios.
Just like real-world systems, it requires a lot of time to develop \acp{sds} with these capabilities, which often makes this option impractical for research.
\Cref{sec:adaptive_spoken_dialogue_systems} discusses further facets and possible solutions for integrating accommodation capabilities into \acp{sds}.

Using these two methods, various experiments have been conducted to measure accommodation in \ac{hci}.
%Various studies have investigated entrainment and priming in \acp{sds}, aiming to better understand \ac{hci} dynamics and improve task-completion rates.
\citet{Bergqvist2020nontrivial, Lopes2011primes}, for example, focused on dynamic entrainment and adaptation on the lexical level and found that users adapt to a system's terminology that differs from theirs.
This also led to improved performance in the given tasks.
\citet{Parent2010lexical} examined the correlation between lexical choices and word frequency using the \emph{Let's Go}~\ac{sds} \citep{Raux2005letsgo} and found that users adapt more to words that occur more often.
While these studies addressed the changes in experimental, scripted scenarios, the theoretical foundations for studying these changes in spontaneous dialogue exist as well \citep{Brennan1996lexical}.
\citet{Gasic2013policy, Levin2000stochastic} provide examples of online adaptation for dialogue policies and strategies.
Noticeably, while all the studies mentioned above examined various facets of dialogues, none of those are related to the auditory aspects of speech -- the primary modality used to interact with \acp{sds} -- but other did:
\citet{Benus2018prosodic} found relationships between the level of users' trust toward an avatar and the degree of the system's vocal entrainment or disentrainment.
Similarly, \citet[][pp.~142-144]{Levitan2014acoustic} shows relationships between prosodic entrainment and how much participants liked the avatar they were interacting with.
\citet{Bell2003prosodic} found that users' speech rate can be manipulated using a human-simulated \ac{sds}.
Similar results were found when intensity changes in children's interaction with synthesized output were examined \citep{Coulston2002amplitude}.
All these experiments focus on \ac{hci}, while those in \cref{sec:linguistic_accommodation} concentrate on \ac{hhi}.
However, accommodation in \ac{hhi} and \ac{hci} has not been directly compared within the same interaction, as done in \cref{chap:speech_variations_in_hhci}.
Furthermore, mainly suprasegmental characteristics have been studied for accommodation in \ac{hci}, mostly due to technical limitations (see \cref{sec:adaptive_spoken_dialogue_systems} for details).
A wizard-of-Oz experiment with a focus on \emph{segmental} features is described in \cref{chap:shadowing_in_sung_music_and_human_computer_interaction}.