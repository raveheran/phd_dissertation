\chapter{Shadowing Experiment Stimuli}
\label{app:shadow_experiment_stimul}

\section*{Recording Fillers}
\begin{enumerate}
	\item Der Schrank wird heute geliefert. (\textit{The cabinet will be delivered today.})
	\item Wo finde ich ein neues Glas. (\textit{Where do I find a new glass?})
	\item Der Markt findet donnerstags statt. (\textit{The market takes place on Thursday.})
	\item Sie wirkt recht gut informiert. (\textit{She seems to be very well informed.})
	\item Ist das der Weg zu dir nach Hause? (\textit{Is this the way to your home?})
\end{enumerate}

\section*{Baseline Fillers}
\begin{enumerate}[resume]
	\item Der Eimer ist aus Plastik. (\textit{The bucket is made of plastik.})
	\item Im Kühlschrank liegt ein Pfirsich. (\textit{There is a peach in the fridge.})
	\item Diese Technik wird noch entwickelt. (\textit{This technique will be further developed.})
	\item Das war sehr höflich von dir. (\textit{That was very nice of you.})
	\item Lena geht heute früher ins Bett. (\textit{Lena goes today early to bed.})
\end{enumerate}

\section*{Experiment Fillers}
\begin{enumerate}[resume]
	\item Die Katze weckt mich immer auf. (\textit{The cat always wakes me up.})
	\item Der Kaffee war ja schon kalt. (\textit{The coffee was already cold.})
	\item Wer fliegt heute in den Urlaub? (\textit{Who flies today on vacation?})
	\item Warum regt er sich denn so auf? (\textit{Why is he so upset?})
	\item Das wird ein schönes Geschenk. (\textit{This will be a pretty present.})
	\item Ich hätte gern zwei kleine Brüder. (\textit{I would gladly have two brothers.})
	\item Das Heft war gestern noch da. (\textit{Yesterday the notebook was still here.})
	\item Die Glühbirne ist leider kaputt. (\textit{Unfortunately the light bulb is  broken.})
	\item Sucht sich Karin eine neue Arbeit? (\textit{Is Karin looking for a new job?})
	\item Wird die Wohnung noch renoviert? (\textit{Will the apartment be renovated?})
	\item Sara hat eine andere Meinung. (\textit{Sara has another opinion.})
	\item Habt ihr das rote Auto erkannt. (\textit{Have you recognized the red car?})
	\item Ich täusche mich so gut wie nie. (\textit{I never delude myself.})
	\item Keiner glaubt diese Geschichte. (\textit{No one believes this story.})
	\item Kommt Fabian auch zu dem Fest. (\textit{Does Fabian come to the festival as well?})
\end{enumerate}

\section*{\textipa{[\c{c}]} vs.\ \textipa{[k]}}
\begin{enumerate}[resume]
	\item Kommt Essig in den Salat? (\textit{Does vinegar come into the salad?})
	\item Der König hält eine Rede. (\textit{The king speaks.})
	\item Kommt Ludwig heute Abend mit? (\textit{Does Ludwig join today evening?})
	\item Es ist ganz schön staubig im Keller. (\textit{It is pretty dusty in the basement.})
	\item Ich bin süchtig nach Schokolade. (\textit{I am addicted to chocolate.})
\end{enumerate}

\section*{\textipa{[e]} vs.\ \textipa{[E]}}
\begin{enumerate}[resume]
	\item Die Bestätigung ist für Tanja. (\textit{The confirmation is for Tanja.})
	\item War das Gerät sehr teuer? (\textit{Was the device very expensive?})
	\item Ich mag die Qualität deiner Tasche. (\textit{I like the quality of your bag.})
	\item Der Schädling sieht aber komisch aus. (\textit{The pest looks funny.})
	\item Wie viel Verspätung hat der Zug? (\textit{How much delay does the train have?})
\end{enumerate}

\section*{\textipa{[@n]} vs.\ \textipa{[\s{n}]}}
\begin{enumerate}[resume]
	\item Sie begleiten dich zur Taufe. (\textit{They are accompanying you to the baptism.})
	\item Wir besuchen euch bald wieder. (\textit{We will visit you soon again.})
	\item Sind die Küchen immer so groß? (\textit{Are the kitchens always so big?})
	\item Wir reden ohne Unterbrechung. (\textit{We are talking without interruption.})
	\item Sind die Affen denn zutraulich? (\textit{Are the monkeys trustful?})
\end{enumerate}
